\section{データ構造}
\subsection{Union-Find木}
\lstinputlisting{data-structure/disjoint_set.cpp}

\subsection{Meldable Heap}
\lstinputlisting{data-structure/meldable_heap.cpp}

\subsection{Binary-Indexed-Tree}
0-indexed
\lstinputlisting{data-structure/binary_indexed_tree.cpp}
\subsection{Segment Tree}
区間addとRMQができる.
\lstinputlisting{data-structure/segtree_range_add.cpp}

\subsection{Sparse table}
\lstinputlisting{data-structure/sparse_table.cpp}

\subsection{赤黒木}
\lstinputlisting{data-structure/rbtree.cpp}
\subsection{永続赤黒木}
\lstinputlisting{data-structure/persistent_rbtree.cpp}
\subsection{wavelet行列}
N := 列の長さ\par
M := 最大値\\
\subsubsection{完備辞書}
\begin{table} [htb]
  \begin{tabular} { |l|l| } \hline
    function & 計算量 \\ \hline
    count & $O(1)$ \\ \hline
    select & $O(\log N)$ \\ \hline
  \end{tabular}
\end{table}
\lstinputlisting{data-structure/fid.cpp}
\subsubsection{wavelet行列}
\begin{table} [htb]
  \begin{tabular} { |l|l|c|c| } \hline
    function & 計算量 & FID::count & FID::select\\ \hline
    count & $O(\log M)$ & o & \\ \hline
    select & $O(\log N \log M)$ & o & o \\ \hline
    get & $O(\log M)$ & o & \\ \hline
    maximum & $O(\log M)$ or $O(k\log M)$ & o & \\ \hline
    kth\_number & $O(\log M)$ & o & \\ \hline
    freq & $O(\log M)$ & o & \\ \hline
    freq\_list & $O(k\log M)$ & o & \\ \hline
    get\_rect & $O(k\log N \log M)$ & o & o \\ \hline
  \end{tabular}
\end{table}
\lstinputlisting{data-structure/wavelet.cpp}
