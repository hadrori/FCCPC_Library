
\section{グラフ}
\lstinputlisting{graph/graph.cpp}
\subsection{強連結成分分解}
\subsubsection{関節点}
$O(E)$\par
ある関節点uがグラフをk個に分割するときartにはk-1個のuが含まれる. 不要な場合はuniqueを忘れないこと.\par
\lstinputlisting{graph/articulation.cpp}

\subsubsection{橋}
$O(V+E)$\par
\lstinputlisting{graph/bridge.cpp}

\subsubsection{強連結成分分解}
$O(V+E)$\par
\lstinputlisting{graph/strongly_connected_component.cpp}

\subsubsection{無向中国人郵便配達問題}
$O(om\log n+o^22^o)$, $-O2$で$o\leq 18$程度が限界
\lstinputlisting{graph/chinese_postman.cpp}
\subsubsection{全点対間最短路(Johnson)}
$O(max(VE\log V, V^2))$
\lstinputlisting{graph/johnson.cpp}
\subsubsection{無向グラフの全域最小カット}
$O(V^3)$
\lstinputlisting{graph/minimum_cut.cpp}

\subsection{フロー}
\subsubsection{最大流}
$O(EV^2)$\par
\lstinputlisting{graph/flow/max_flow.cpp}

\subsubsection{二部マッチング}
$O(EV)$\par

\lstinputlisting{graph/flow/bi_matching.cpp}
\subsubsection{最小費用流}
$O(FE\log V)$\par
\lstinputlisting{graph/flow/minimum_cost_flow.cpp}

\subsubsection{Gomory-Hu木}
$O(V MAXFLOW)$
\lstinputlisting{graph/flow/gomory_hu.cpp}

\subsection{木}
\subsubsection{木の直径}
ある点(どこでもよい)から一番遠い点aを求める. 点aから一番遠い点までの距離がその木の直径になる.\par
\subsubsection{最小全域木}
\lstinputlisting{graph/minimum_spanning_tree.cpp}
\subsubsection{最小全域有向木}
$O(VE)$
\lstinputlisting{graph/ms_arborescence.cpp}
\subsubsection{最小シュタイナー木}
$O(4^{|T|}V)$ \par
gは無向グラフの隣接行列. Tは使いたい頂点の集合.\par
\lstinputlisting{graph/minimum_steiner_tree.cpp}
\subsubsection{木の同型性判定}
順序付き$O(n)$\par
順序なし$O(n\log n)$
\lstinputlisting{graph/tree_isomorphism.cpp}

\subsection{包除原理}
\subsubsection{彩色数}
$O(2^VV)$\par
N[i] := iと隣接する頂点の集合(iも含む)\par
\lstinputlisting{graph/graph_coloring.cpp}
