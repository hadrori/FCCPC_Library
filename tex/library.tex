\documentclass[9pt,twocolumn,a4paper,landscape]{extarticle}
%
\usepackage[dvipdfmx]{graphicx}
\usepackage{latexsym}
\usepackage{amsmath,amssymb}
\usepackage[cmbtt]{bold-extra}
\usepackage{bm}
\usepackage{graphicx}
\usepackage{ascmac}
\usepackage{indentfirst}
%
\setlength{\columnsep}{3zw} 
\setlength{\topmargin}{20mm}
\addtolength{\textheight}{-40mm} 
\addtolength{\topmargin}{-2in}
\setlength{\textheight}{190mm}
\setlength{\oddsidemargin}{-15mm}
\setlength{\textwidth}{279mm}
%
\newcommand{\divergence}{\mathrm{div}\,}  %ダイバージェンス
\newcommand{\grad}{\mathrm{grad}\,}  %グラディエント
\newcommand{\rot}{\mathrm{rot}\,}  %ローテーション
%
%しおり
\setcounter{tocdepth}{3}
\usepackage[dvipdfmx,
bookmarks=true,
anchorcolor=blue,
linkcolor=blue,
urlcolor=blue,
colorlinks=true,
bookmarksopen=true,
]{hyperref}


% Listingsの設定
\usepackage{ascmac,here,txfonts,txfonts}
\usepackage{listings,jlisting}
\usepackage[dvips]{color}
\lstset{
  breaklines = true,
  language=c++,
  captionpos=b,
  basicstyle={\ttfamily \footnotesize},
  commentstyle={\itshape \color[cmyk]{1,0.4,1,0}},
  keywordstyle={\ttfamily \color[cmyk]{0.3,0.9,0,0}},
  stringstyle={\ttfamily \color[rgb]{0.8,0,0}},
  frame=tlrb,
  framesep=5pt,
  showstringspaces=false,
  numbers=left,
  stepnumber=1,
  numberstyle=\tiny,
  tabsize=2,
  xleftmargin=5mm, 
}
\usepackage{fancyhdr}
\pagestyle{fancy}
\lhead{\empty}
\chead{\empty}
\rhead{\empty}
\lfoot{FCCPC Library}
\cfoot{\empty}
\rfoot{\thepage}
\renewcommand{\footrulewidth}{0.1pt}
\renewcommand{\headrulewidth}{0pt}
\footskip 24pt
%
\begin{document}
\tableofcontents
\newpage
%
%
\section{準備}
\subsection{init.el}
linumはemacs24のみ\par
\lstinputlisting[language=Lisp]{./code/init.el}

\subsection{tpl.cpp}
\lstinputlisting{./code/tpl.cpp}

\section{文字列}
\subsection{Aho-Corasick法}
$O(N+M)$\par
\lstinputlisting{./code/aho_corasick.cpp}

\section{グラフ}
\subsection{強連結成分分解}
\subsubsection{関節点}
$O(E)$\par
ある関節点uがグラフをk個に分割するときartにはk-1個のuが含まれる. 不要な場合はuniqueを忘れないこと.\par
\lstinputlisting{./code/art_point.cpp}

\subsubsection{橋}
$O(V+E)$\par
\lstinputlisting{./code/bridge.cpp}

\subsubsection{強連結成分分解}
$O(V+E)$\par
\lstinputlisting{./code/scc.cpp}

\subsection{フロー}
\subsubsection{最大流}
$O(EV^2)$\par
\lstinputlisting{./code/max_flow.cpp}

\subsubsection{二部マッチング}
$O(EV)$\par
\lstinputlisting{./code/bi_matching.cpp}

\subsubsection{最小費用流}
$O(FE\log{V})$\par
\lstinputlisting{./code/min_cost_flow.cpp}

\subsection{最小シュタイナー木}
$O(4^{|T|}V)$ \par
gは無向グラフの隣接行列. Tは使いたい頂点の集合.\par
\lstinputlisting{./code/min_steiner_tree.cpp}

\end{document}
