\documentclass{jsarticle}
\setlength{\textheight}{252mm}
\setlength{\topmargin}{-20mm}
\parindent=0pt
\pagestyle{empty}

\begin{document}

\twocolumn[\section*{Verdict別チェックリスト}]

\vbox{}\subsection*{Wrong Answer}

\begin{itemize}
\item 初期化したか? またそれは適切か?
  \begin{itemize}
  \item とくに複数テストケースのとき
  \end{itemize}
\item オーバーフローの危険性はないか?
  \begin{itemize}
  \item \verb|1LL << n|
  \end{itemize}
\item 境界条件は正しいか?
\item 出力形式は正しいか?
\item 解なしの扱いは正しいか?
  \begin{itemize}
  \item 処理を抜けているかにも注意
  \end{itemize}
\item タイプミスはないか?
\item 入力を破壊して,出力に影響していないか?
\item コーナーケースはないか?
\item 計算誤差は許容範囲内か?
  \begin{itemize}
  \item \verb|inf|, \verb|eps|は大小比較でしか使わない
  \end{itemize}
\item 使っていない変数はないか?
\item DPの更新順は正しいか?
\end{itemize}

\subsection*{Time Limit Exceeded}

\begin{itemize}
\item 計算量は許容範囲内か?
  \begin{itemize}
  \item 複数テストケースに注意
  \end{itemize}
\item 入出力はボトルネックにならないか?
\item メモ化を忘れていないか?
\item 無限ループの危険性はないか?
\item スターグラフなど,極端なケースでも大丈夫か?
\item ボトルネックを定数倍高速化できないか?
  \begin{itemize}
  \item $\verb|int|<\verb|long long|<\verb|double|$
  \item ループアンローリングしてみる(奥の手)
  \end{itemize}
\item 埋め込めるところはないか?(奥の手)
\end{itemize}

\subsection*{Runtime Error\,/\,Segmentation Fault}

\begin{itemize}
\item 配列やコンテナのサイズは適切か?
\item 範囲外アクセスの危険性はないか?
\item ポインタは\verb|NULL|初期化されているか?
\item 入力は正しく受け取れているか?
  \begin{itemize}
  \item \verb|getline|の前に\verb|cin.ignore()|
  \end{itemize}
\item スタックオーバーフローの危険性はないか?
  \begin{itemize}
  \item 再帰が深くなることはないか?
  \item 訪問済みフラグを立て忘れていないか?
  \end{itemize}
\end{itemize}

\subsection*{出力が壊滅的なとき}

\begin{itemize}
\item \verb|return|を忘れていないか?
\item $1$-originを直し忘れていないか?
\item 添字に間違いはないか?
\item 変数名は衝突していないか?
\item 演算子の優先順位に間違いはないか?
  \begin{itemize}
  \item \verb|(1 << n) - 1|, \verb|(x & y) == z|
  \end{itemize}
\end{itemize}

\subsection*{そもそも解けないとき}

\begin{itemize}
\item 題意把握に間違いはないか?
\item 制約条件を見逃していないか?
\end{itemize}

\subsection*{解法メモ}

\begin{description}
\item[辞書順最小の解] 解の存在判定
\item[集合の$2$-分割] 最小カット
\item[重複が$k$個以下の区間] $k$-最小費用流
\item[幾何] 候補の離散化
\item[高速化] deque,segtree, doubling, Monge性
  \begin{itemize}
  \item 自明DPの漸化式を書いてみる
  \item bool DPは単調性を疑ってみる
  \end{itemize}
\item[とりあえず]\mbox{}
  \begin{itemize}
  \item 二分探索できないか?
  \item ソートしたら解けないか?
    \begin{itemize}
    \item 「いい順序」はあるか?
    \end{itemize}
  \item 逆順\,/\,補集合\,/\,双対を考えてみる
    \begin{itemize}
    \item フローだと思ったらカットでも
    \end{itemize}
  \item 式を変形してみる
  \item LP定式化してみる
  \end{itemize}
\end{description}

\subsection*{デバグメモ}

空に向かってでもいい,一行一行説明してみよう!

\end{document}
